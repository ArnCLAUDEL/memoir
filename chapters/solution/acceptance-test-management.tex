\section{Acceptance Test Management}\label{sec:acceptance-test-management}

This section will cover the acceptance test management, mainly regarding the
scenarios, on how to write, sort and execute them.

\subsection{Cucumber scenario format}\label{subsec:cucumber-scenario-management}
% TODO some advice on cucumber scenario format

\subsubsection{Core scenario}
% TODO homemade
% TODO interact with the database, web services etc.
% TODO describe busniess rule

\subsubsection{UI scenario}
% TODO we can use NoraUI

\subsubsection{API scenario}
% TODO we can improve NoraUI for the API testing

\subsection{Scenario Tags}\label{subsec:scenario-tags}
Cucumber tag is a very helpful feature that allow us to classify our
scenarios into various categories.
This non-exhaustive section is going to present several relevant categories.

\subsubsection{Component Under Test}
We already shown an example of this tag, with the UI, API and Core scenarios.
They can be tagged using \textit{@UI}, \textit{@API} or \textit{@Core} to
indicate that they only test a specific type of component.
Hence it would be possible to easily execute the test only for the UI, which
would avoid to execute everything at the same time.

\subsubsection{Business Domain}
A tag can be added to each scenario to indicate which business domain they
belong to.
First, just like the previous tag, it allows us to execute the test of
specific business domain.
But mostly, it will serve as a hint for the documentation.

Scenarios are the living the documentation of the application.
However, when there are a lot of scenarios, it can be hard to navigate
between them and immediately understand which business domain is involved.
Hence, adding more tags like \textit{@Mailing} or \textit{@Authentication}
can be helpful.

\subsubsection{Execution Speed}
Some scenarios can be very long, when for example they involve asynchronous
computation or generate a lot of data.
These tests will inevitably slow down the entire test process and delay the
test report of the faster ones.
It's therefore interesting to add a hint about the execution speed or
resource usage like \textit{@Slow} or \textit{@Heavy}.
Their execution can be delayed so they don't block fast tests to be executed.

\subsubsection{Manual Test}
As we said earlier in the previous section of this chapter, not all the tests
can be automated, even if they can be written as a scenario.
However it's not necessarily useless to write them as a scenario, because
automation is not the only advantage of this format.
It will still contribute to the living documentation and describe the step to
do the test, using the domain vocabulary and a standard format.

Hence, manual test written as a scenario can be tagged with a
\textit{@Manual}, so we can exclude them from the automation process but
still include them as a normal scenario in the application.

\subsubsection{Current release}
Lastly, we can tag the scenario of the features that will be released in the
current version.
It will serve as an overview of what will be added and allows us to
only execute the tests of the current version, which would reduce the
execution time and feedback duration on the hot features.

\subsection{Scenario lifecycle}\label{subsec:scenario-lifecycle}
% TODO the lifecycle between writing and validation

\subsubsection{Submitted}
% TODO @TODO

\subsubsection{Assigned}
% TODO @WIP

\subsubsection{Pending Review}
% TODO @Review

\subsection{Test execution}\label{subsec:test-execution}
When there are a lot of test, especially UI tests, test execution can quickly
get longer and longer.
Slow tests mean on one hand longer feedback but also more costly as the CI
platform will be heavily solicited and the team is likely to waste time waiting
for the build to finish.
This is why we are going to propose a new test execution strategy.
% TODO test can be long to execute

\subsubsection{Execution Strategy}
% TODO some test are more likely to fail than others
% TODO local test every minute
% TODO integration test every 10 minutes
% TODO module test every 100 minutes
% TODO appplication test every 1000 minutes
% TODO recreate environment from scratch

\subsubsection{Time vs Resources}
% TODO trade feedback duration against CI resources