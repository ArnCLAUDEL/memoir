\section{Software Craftsmanship}\label{sec:software-craftsmanship}
Software Craftsmanship is a culture carried by the most passioned developers,
Software Crafters.
There is a great manifesto that describes perfectly the main ideas.

\begin{quotation}
As aspiring Software Craftsmen we are raising the bar of professional
software development by practicing it and helping others learn the craft. \\
\newline
Through this work we have come to value : \\
\newline
Not only working software, \\
\hspace*{1cm} but also \textbf{well-crafted software}\\
Not only responding to change, \\
\hspace*{1cm} but also \textbf{steadily adding value} \\
Not only individuals and interactions, \\
\hspace*{1cm} but also a \textbf{community of professionals} \\
Not only customer collaboration, \\
\hspace*{1cm} but also \textbf{productive partnerships} \\
\newline
That is, in pursuit of the items on the left we have found the items
on the right to be indispensable. \\
\newline
\small{\textcopyright 2009, the undersigned.
this statement may be freely copied in any form,
but only in its entirety through this notice.}
\end{quotation}
The two most important points are Quality and Community. \\
Software Craftsmanship's goal is to deliver better software, by using the
most advanced methods in order to reach a very high level of quality.
Crafters always keep in my minds simple principles and rules
(that we'll cover later in the Quality section) when developing, so they
maintain the quality level high while the important business value to their
projects.\\
\newline
Both the software and the crafter are in a continuous improvement process.
This is where the community aspect is the key because it's hard (if not
impossible) to learn and improve its skills alone.
Therefore, Mentoring is at the heart of the culture and knowledge sharing is
part of the daily tasks of a crafter. \\
\newline
There are a lot of books written by crafters, on various methodologies,
programming techniques or design principles.
Blogs are also very popular where, for example, blogers explain how they
used a particular method on their project, how it brought extra value to
the team or even the pitfalls, explaining what went wrong and why.
It's very valuable to have so much feedback, from a lot of different
projects, teams and business domains. \\
\newline
Events are very frequent too, from summit conferences to small talks or
code dojo.
The main idea is still to share knowledge, learn from each others and keep
the continuous improvement process on.
For example, you can face an issue at work or try a product and no body
in your team have a lot of experience with it.
Then you could attend to an event organized by the community of this product,
there you will find a lof of experienced user, see how they use it and even
talk about your specific problem to them.
You would then share what you learnt to your team the next morning and
improve both your project and team knowledge. \\
\newline
We could have a much more longer presentation of what make Software
Craftsmanship but we already have the key features for our problematic. \\
A focus on Quality and the will to learn and share knowledge to keep
improving the overall Quality of the project.
