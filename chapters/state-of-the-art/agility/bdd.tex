\section{Behavior Driven Development}\label{sec:bdd}
%http://kilianwebservice.de:8080/documents/20195/62637/BDD_in_Action.pdf/3a6a2aef-5bf5-4ccd-a33d-e153d1a9edb3
%https://dannorth.net/introducing-bdd/

This section will introduce the notion of \textit{Behaviour Driven
Development}, with its key concepts, benefits and limits or pitfalls.
This section is mostly based on the book \textit{BDD in Action}.

\subsection{Key concepts}\label{subsec:key-concepts}

\subsubsection{A new version of TDD}
We already mentioned TDD while presenting Extreme Programming, this method
basically consist of writing a failing test before writing any production code.
Well few years after TDD, Dan North wrote an article to introduce BDD\@.

Traditional unit tests written TDD are very low-level and focus primarily on
a class or method.
They were written to test the possible path of a method and were therefore very
tied to the implementation.
Dan talked about a problem he used to encounter when he was teaching TDD in
a teams.
People were a bit lost with TDD, on how to start, how to name its tests or how
even to know when there is enough test for this feature.

He decided to introduce some new practices and conventions to focus on the
behaviour when writing test, instead of focusing on the implementation itself.
This means that a test should simply describe what the method \textit{should}
do instead of \textit{how}.

\subsubsection{Focus on valuable feature}
In traditional methods such as waterfall, the integration team starts by
understanding all the customer needs and then designs a solution that should
fulfill the needs.

When the design is done, the developers will just have to implement a bunch
of requirements, which when developed will shape the features of the software.
The problem is that the end user is not really involved in the process, and
nothing takes into account what the end user might expect when using the
application.

In addition of making the needs harder to understand, this can cause the
development of completely useless features, that were in the design model but
won't be used at all by the users.
BDD tries to avoid this by bringing the end users at the center of the
discussion.

The idea is to discuss about the expected behaviour for an end user, and
progressively build the features by only adding the business value.
This will also help the developers to really understand what they're
implementing and see how this feature will make the current software more
valuable.

\subsubsection{Work together}
% TODO collaborative practice between BA, dev and testers
% TODO stop to written requirements that need to be understood
% TODO don't tell what they need, let them design the solution
% TODO common sens of ownership

\subsubsection{Embrace uncertainty}
% TODO -- remove this section if it's too long
% TODO assume that the requirements will change
% TODO early feedback as much as possible
% TODO feature prioritization

\subsubsection{Features with concrete examples}
% TODO help to define and understand the feature
% TODO share a common vocabulary
% TODO old written specs are easily misunderstood
% TODO feature exploration is great, especially for testing

\subsubsection{Executable specifications}
% TODO tells objectively when the feature is done
% TODO serves as guideline
% TODO one it works, it turns immediately as a non regression test

\subsubsection{Living documentation}
% TODO always up-to-date
% TODO easy to read and better for newcomers
% TODO examples are better than pages of specification
% TODO easier for the maintainers
% TODO easier to change an existing feature
% TODO fix or delete a broken scenario

\subsection{Benefits}\label{subsec:benefits}
% TODO reduced cost by only adding relevant features
% TODO easier to change an existing feature
% TODO automated tests

\subsection{Limits}\label{subsec:limits}
% TODO requires high collaboration
% TODO works better in agile
% TODO doesn't work well in silo because of the poor communication
% TODO doesn't work well because QA comes late
% TODO can be costly if not written correctly