\section{Why not agility ?}\label{sec:why-not-agility}

We now have an overview of the current methods and tools to improve the quality of our projects. 
We've seen the traditional workflow in both waterfall and V-model, and their limitations.
We've also covered some agile methodologies that get rid of certains of these problems.

An easy solution could be simply abandon the waterfall or V-model and become agile.
However there's no silver bullet and sometimes, switching from waterfall to agile is not necessarily the solution. 
This section is going to cover some of the reasons to stay in with the waterfool methodology.

\subsection{Team members}\label{subsec:team-members}
In a team where members have been working in waterfall for decades, switching to a new methodology can be tricky.
Whether it's because of changes' fear or reticence against the trending agility methodology, people may be reluctant to do the transition.
When people are almost forced to change their habits, the transition never completely occurs and remains partly done in the long term.
In this situation, it's likely to be worst than if the team stayed with its traditional methodlogy.
This is the case for team members but it's even more true for clients.

\subsection{Clear requirements}\label{subsec:clear-requirements}
One of the main statements of agility is that the requirements are usually unclear and likely to change.
Therefore in an agile context, the analysts will spend more time talking with the client to understand its needs.
An real work of requirements exploration is made, to help the client formalize its desires.

However, this kind of exploration is not always needed to define the requirements.
For example, public services or financial agencies etc.
are not going to invent or discuss their needs with an analyst.
Tax rates, invoices format or government rules are already clearly defined.

\subsection{Project context}\label{subsec:project-context}
Some project contexts may also not be well suited for agile and its iterative cycle.
A client may not want frequent releases of its software, but rather few releases in a year.
it can be because the terminal cannot be easily updated or even because there's not that much requirement changes.
The client may also not have the time or the resources to be really involved in the project.
Small one-shot applications are clearly not suited for agile and will be cheaper in a waterfall methodology.
 
The importance of documentation can also matter.
In an agile context, we tends to get rid of the documentation, for good reasons.
Whereas in waterfall, documentation is contractual and therefore part of the delivry content.
Some clients may feel more comfortable when they have this documentation, with all the supports that helped in making it.
