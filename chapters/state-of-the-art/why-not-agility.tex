\section{Why not agility ?}\label{sec:why-not-agility}
We now have an overview of the current methods and tools to improve the
quality of our projects.
We've seen the traditional workflow in both waterfall and V-model, and their
limitations.
We've also covered some agile methodologies that get rid of certain of these
problems.

An easy solution could be simply to abandon the waterfall or V-model and become
agile.
However agile, like any other method, is not a silver bullet and sometimes,
switching from waterfall to agile is not necessarily the solution.
This section is going to cover some of the reasons to stay in with the
waterfall methodology.

\subsubsection{Team members}
In a team where members have been working in waterfall for decades, switching
to a new methodology can be tricky.
Whether it's because of the fear of change or reticence against the
trending agility methodology, people may be reluctant to do the transition.

When people are almost forced to change their habits, the full transition
almost never occurs and the team keeps some of its old habits.
In this situation, it's likely to be worst than if the team stayed with its
traditional methodology
This is the case for team members but it's even more true for clients.

\subsubsection{Clear requirements}
One of the main statements of agility is that the requirements are usually
unclear and likely to change.
Therefore in an agile context, the analysts will take the time to talk with
the client to understand its needs.
A real work of requirements exploration is made, to help the client
formalize its desires.

However, this kind of exploration is not always needed to define the
requirements.
For example, public services or financial agencies etc.
are not going to invent or discuss their needs with an analyst.
Tax rates, invoices format or government rules are already clearly defined.
The integration team just have to retrieve all these formalized business rules.

\subsubsection{Project context}
Some project contexts may also not be well suited for agile and its iterative
cycle.
A client may not want frequent releases of its software, but rather few
releases in a year.
Whether it can be because software updates are expensive, which is common in
legacy systems, or simply because there's not that much requirement changes.

The client may also not have the time or the resources to be really involved
in the project.
Agile project take the time to make a lot of demo, get early feedback from
the client etc.
which is a great approach.
But when the client only have a small one-shot application, with predefined
user interface and features, then agility is not well suited.
Especially when the integration team works in remote, there's usually not a
lot of communication with the client during the development process.

The importance of documentation can also matter.
In an agile context, we tend to get rid of the documentation, for relevant
reasons.
Whereas in waterfall, documentation is contractual and therefore part of the
delivery content.
Some clients may feel more comfortable when they have this documentation,
with all the supports that helped in making it.
