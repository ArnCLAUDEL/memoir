\section{Detailed Problematic}\label{sec:detailed-problematic}
%http://softwaretestingfundamentals.com/software-quality/
There are a lot of possible definitions for Quality in software projects, for
this problematic, we've chosen the one from the \textit{International Software
Testing Qualifications Board} :
\begin{quote}
The degree to which a component, system or process meets specified
requirements and/or user/customer needs and expectations.
\end{quote}
This definition is very abstract so we're going to detail what exactly is the
difference between a requirement, a need and an expectation.

%https://www.iiba.org/standards-and-resources/glossary/
%http://mohamedelgendy.com/blog/business-needs-vs-requirements.html
%https://westwindconsulting.com/Requirements%20vs%20expectations.html
According to the \textit{International Institute of Business Analysis}, a
requirement is a description of the need of a stakeholder that must be met in
order to achieve the business requirements.
They may serve as a bridge between business requirements and the various
categories of solution requirements.
A need is a problem of strategic importance to be addressed and could be
described as a high-level representation of the requirement.

These two terms refers to what the stakeholders express to the integration
team and it is (supposed to be) well defined.
However, an expectation is usually more unclear and harder to state because
it tends to be more subjective and is often implicit.
They may have different shape and be at different level of the project such as
people interaction, process, tools or code.


Let's list some examples. \\
One might expect a feature to be easily added to the existing application,
without breaking any existing features while integrating perfectly with all the
common features already implemented.
These features should be easy to update when needed and, again, shouldn't
affect any other components in the project.
The requirements should be covered by strong, replayable and trustful tests,
that should not break on every modification and that allow us to quickly
detect regression and bugs.

There're also examples with unexpected error handling, performances,
robustness etc.
We can find a lot of examples but these simple ones are enough for our
problematic.


All these kind of expectations, that one could expect for any project, are
not mandatory to make a working software.
Therefore, they can be classified as quality details and omitted on certain
projects, because they don't directly add business value to the software.
The problem is that every ignored detail will slowly but constantly degrades
the project's health.
It might not be noticeable in the short term, however the consequences will
show up after few weeks or months and will overall slow down every move in
the project.

This memoir will present the state of the art in term of tool
and process that help improving the quality of a project and address quality
detail issues. \\
We'll highlight the residual problems and also point out some new ones that
we'll cover in our solution.