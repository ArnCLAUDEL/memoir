\section{Detailed Problematic}\label{sec:detailed-problematic}
%http://softwaretestingfundamentals.com/software-quality/
There are a lot of possible definitions for Quality in software projects, for
this problematic, we've chosen the one from the \textit{International Software
Testing Qualifications Board} :
\begin{quote}
The degree to which a component, system or process meets specified
requirements and/or user/customer needs and expectations.
\end{quote}
This definition is very abstract so we're going to detail what exactly is the
difference between a requirement, a need and an expectation. \\
%https://www.iiba.org/standards-and-resources/glossary/
%http://mohamedelgendy.com/blog/business-needs-vs-requirements.html
%https://westwindconsulting.com/Requirements%20vs%20expectations.html
According to the \textit{International Institute of Business Analysis}, a
requirement is a description of the need of a stakeholder that must be met in
order to achieve the business requirements.
They may serve as a bridge between business requirements and the various
categories of solution requirements.
A need is a problem of strategic importance to be addressed and could be
described as a high-level representation of the requirement. \\
These two terms refers to what the stakeholders express to the integration
team and it is (supposed to be) well defined.
However, an expectation is usually more unclear and harder to state because
it tends to be more subjective and is often implicit.
For example, a stakeholder might expect its application to run smoothly, no
matter how many users it has, or that sensitive data is well secured and
compliant with the current legislation.
He won't explicitly require that its application is reliable, fast, secure
and easy to adapt, because he could think that it's just the standard way to go.
\\
Quality details may have different shape and be at different level of the
project such as people interaction, process, tools and even production code. \\
\newline
Projects that don't pay enough attention to the details, will see their
technical debt growing bigger and bigger.
The consequence of high technical debt is generally a dramatic
cost increasing of every move in the project. \\
Let's show a quick example, we'll take a project that is meeting all the
requirements of the stakeholder and show some details (that don't prevent
the project features from working) that can slow down the project lifecycle
and occurs extra cost. \\
\newline
We have 1--2 business persons and 2--3 developers, the communication between
the two teams is weak and the needs/requirements are specified around
coffees, mails and sometimes on slides but nothing formal.
The developers interpret the feature and quickly implement it with an extensive
use of copy paste.
It's integrated and manually tested, also very quickly, and then the cycle
is repeated until all the features are implemented. \\
The business is then able to test the whole system and a lot of round trips
are made to correct the bugs and misunderstanding of the specs.
This part takes a lot longer than the development part, because developers
now need to dig again in the code and understand what they've written in
order to fix the bug.
They obviously forgot what they did in the first place, how they tested it
and they end up re creating the whole feature again.
In addition, the fix is likely to silently break other features, that were
previously tested and validated.
We can choose to run all the tests again for each round trip, but at a very
expensive cost or we can choose to accept the risk of introducing a bug into
production.
For now, we'll suppose that there's no new bug.
After this testing phase, all requirements are validated, the application is
delivered and all these steps will be repeated again for the next incoming
version. \\
\newline
At this point, we have a working software, but how about the quality ? \\
If we refer to our definition, a high quality software should meet the needs,
requirements and expectations of the stakeholder.
Needs and requirements are fulfilled, but what about expectations ? \\
Why is there so many bug or divergence with the specification ? \\
Why does it take so long to modify a small behaviour present in almost every
features ? \\
Why shall we run already passed tests again when another feature is modified
? \\
Why is it so hard to fix a bug whereas the feature was written very quickly ? \\
Where are the proof of each tests ? \\
Which test have been run ?
And when ?
On which version ? \\
\newline
All these kind of expectations, that one could expect for any project, are
not mandatory to make a working software.
Therefore, they can be classified as quality details and omitted on certain
projects, because they don't directly add business value to the software.
The problem is that every ignored details will slowly and constantly make
things harder and more costly overall. \\
\newline
This memoir will present the state of the art in term of tool
and process that help improving the quality of a project.
We'll highlight the residual problems and propose a solution to address them.