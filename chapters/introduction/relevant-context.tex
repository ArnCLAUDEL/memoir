\section{Relevant Context}\label{sec:relevant-context}
This section describes the relevant project contexts for this problematic
and its solution.

The \textit{domain} of the project should be a non-critical one.
By non-critical, we reference any project where there is no human life,
financial market, aerospace system and any other "critical domain" at
the core of the application.

Of course, out \textit{methodology} will be Waterfall, or equivalent.
A document-oriented approach, with clear customer needs and long iterations.

Regarding the \textit{architecture} of project, we'll handle a project with
Web UI and an exposed API\@.
Having multiples endpoints may requires multiples tools to specify, develop,
test each of them and they must be managed in order to easily maintain them.

The \textit{team composition} of our context will be made of 3 group of people:
clients, business analyst and developers.
Business analysts define the requirements and functional tests with the client
and then gives the specifications to the dev team.
Clients and business analysts aren't expected to have any particular technical
skills and developers aren't expected to have a deep understanding of the
business domain.

The \textit{programming language} used is not important and we'll try to stay
agnostic but code samples will be written in Java.
Most of the tools and technologies are compatible with the common programming
languages, such as Java, Python, Javascript etc.