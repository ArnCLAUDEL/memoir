This introduction will detail the problematic by defining the Quality in
a software project and what low-quality projects are exposed to.
We will also define the relevant project contexts, that is where the proposed
solution is suitable.
Finally, we will justify the choice of this problematic by describing the
current context in my company and my personal experiences in the area of
Quality and Testing.

\section{Detailed Problematic}\label{sec:detailed-problematic}

%http://softwaretestingfundamentals.com/software-quality/
There are a lot of possible definitions for Quality in software project, for
this problematic, we've chosen the one from The International Software
Testing Qualifications Board :
\begin{quote}
The degree to which a component, system or process meets specified
requirements and/or user/customer needs and expectations.
\end{quote}
This definition is very concise so we're going to detail what exactly is the
difference between a requirement, a need and an expectation. \\
%https://www.iiba.org/standards-and-resources/glossary/
%http://mohamedelgendy.com/blog/business-needs-vs-requirements.html
%https://westwindconsulting.com/Requirements%20vs%20expectations.html
According to IIBA, a requirement is a description of the need of a
stakeholder that must be met in order to achieve the business requirements.
They may serve as a bridge between business requirements and the various
categories of solution requirements.
A need is a problem of strategic importance to be addressed and could be
described as a high-level representation of the requirement. \\
These two terms refers to what the stakeholders express to the integration
team and it is (supposed to be) well defined.
However, an expectation is usually more unclear and harder to state because
it tends to be more subjective.
For example, a stakeholder might expect its application to run smoothly, no
matter how many users it has.
Or that sensitive data is well secured and compliant to the current legislation.
And obviously, that introducing new features to the application shouldn't affect
other existing features. \\
It appears that expectations differ a lot across people and projects, this
why we insist on it because we state that Quality, in a software project, goes
further than meeting the requirements and should concern all the
tiny details that can make the difference in the long run.
Quality details are at every level of the project, from people interaction,
process, tools or even production code. \\
\newline
% TODO risk for low-quality projects


\section{Relevant Context}\label{sec:relevant-context}
This section describes which contexts are relevant to this problematic
and where the proposed solution is suitable. \\
This problematic is common on non-critical IT projects using
programming languages such as Java, Javascript, Python etc.
By non-critical, we reference any project where there is no human life,
financial market, aerospace system and any other "critical domain" at
the core of the application. \\
There are often a Web UI and an exposed API for the end users.
Each of these interfaces have their dedicated tools for development and
testing.
The management of all the different tools can be quite tricky and
directly impacts the cost of maintenance. \\
These projects can have multiples teams that work on, such as business,
dev or testing.
Some teams usually don't have technical skills, like business or even the
testing team.
The communication between them is often hard and sometimes there is
almost no communication. \\
The methodology used, either Agile or Waterfall, does not matter.
Usually, an Agile project is more likely to follow good practices in
Quality and Testing but this is not necessarily the case.
The proposed solution is not Agile-oriented and will work on Waterfall
projects. \\
Finally, as Quality has a cost, low-quality projects usually have smaller
budget.
Therefore the proposed solution only use community or free licences tools
and try to reuse at most as possible the same elements, tools, data etc.

\section{Justification}\label{sec:justification}
This section explains why I have chosen this problematic by describing
my current project, important experiences and my personal interests. \\
I've been working at Sopra Steria for one year, on the project
Chorus Pro, for the Agence pour l'Informatique Financi\`{e}re de l'Etat
(AIFE).
I was in internship in 2018 and I started my apprenticeship in the exact
same team.
It's a big project that handles digital invoices sent by companies to the
State.
It's mostly written in Java, with a Web Portal, REST / Soap API, EDI
exchanges and Spring Batch.
There are a lot of modules, components, environments, teams and of course
a lot of legacy code. \\
Until very recently, there was almost no unit / integration tests, it was
barely manual testing in dev environments and manual functional testing
in staging environment.
The consequences of the lack of testing were obvious and I realized why
testing is so important, the time it can save or the bugs it can prevent.
I saw some people reluctant to add new tests, to automate them or to
improve code quality and it appears that it really is a vicious circle.
\\
\newline
Beside my project where I understood why Quality and Testing are
important, I also attended to, in my opinion the best class of my
training, the Agile class by M. Zam.
This class was a very broad overview of methods, techniques, tools and a
lot of buzzwords that help to create better software.
I would rename this class as Software Craftsmanship, because it is a
bunch of good practices and real life complex methods that are usually
used only by Software Crafters. \\
I was very sceptical by what he showed us, because it seemed very complex
and not natural at all.
Things such as TDD, Pair Programming, DDD were completely different of
what I was used to.
I couldn't understand why people were talking so much about that, how can
people could even work with these methods. \\
So I tried to see what others do, and the best way I found was Meetup.
Meetup are small events, usually with several talks but sometimes you
simply code with other people and you make some tricky exercises or
discover new technologies.
I realized that these unusual methods were actually very common and that
it only depends on you to work with the people you want to. \\
\newline
This problematic was perfectly suited for me. \\
First of all, I'm very interesting in these Quality and Testing concerns,
so I can learn new things, methods, tools about it and see how common
problems are tackled. \\
Second, I was a bit tired of working on the same thing in my project and
I heard that they wanted to work on test automation.
So this problematic could help me to switch to a fresh mission that keeps
me in the mood. \\
Finally, I think that it will be very helpful for me later, in my future
job.
I can learn a lot of new methods and technologies that are used in the
real world and be better prepared to work with.
