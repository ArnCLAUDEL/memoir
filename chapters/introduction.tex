This introduction will detail the problematic and show why there are
still projects with a low level of quality with almost no test.
We will also define the relevant contexts where the proposed solution
of this problematic is suitable.
Finally, I will justify the choice of this problematic by describing the
current context in my company and my personal experiences in the area of
Quality and Testing.
\section{Detailed Problematic}\label{sec:detailed-problematic}
There are a wide variety of IT projects, with different programming
languages, methodologies and business domains.
Practices change frequently over time, new technologies arise,
some disappear after several years, some are broadly used and are even older
than the programmers that use them. \\
Quality and Testing concerns are not new.
Prior to 2000, there was already projects with a very high level of quality
and strong automated tests, but it was very uncommon at this time.
In the early 2000's, a lot of new methodologies and tools emerged in order
to solve problems that programmers used to encounter. \\
In the 2010's, these methods and tools became mature and were in a
lot of projects.
More and more blogs, conferences or training were created to present them,
to show how to use them and introduce them in a legacy project.
The community grew and now a day they are almost a standard in new mainstream
projects. \\
\newline
However, even with the great enthusiasm of the community, there are still
projects where programmers, managers and clients are not convinced by
these methodologies and technologies and keep doing it the old way. \\
Poor design, no refactoring nor code reviews, no unit testing, minimum
manual functional tests, heterogeneous environments etc.
The list can be very long, so do for the consequences.
In the worst case, people give up and accept these bad practices so they
integrate them as part of the normal process. \\
Sometimes, it's hard to see the value of a tool that, at first glance, only
cost a lot of
efforts, time and therefore money, for no extra business value.
The benefits in a long run are generally ignored or under estimated.
The change of the team's habits does have a cost, requires new trainings,
new processes, some extra management for the new methodologies and
extra maintenance for the new tools.
But this is a necessary mean, an investment for the long run in order to
increase the overall quality and prevent the project from disaster. \\
\newline
The solution of this problematic starts therefore by correctly understanding
the features of existing methods and tools and then show how to easily
integrate them all into a project.

\section{Relevant Context}\label{sec:relevant-context}
This section describes which contexts are relevant to this problematic and
where the proposed solution is suitable. \\
This problematic is common on non-critical IT projects using
programming languages such as Java, Javascript, Python etc.
By non-critical, we reference any project where there is no human life,
financial market, aerospace system and any other "critical domain" at the
core of the application. \\
There are often a Web UI and an exposed API for the end users.
Each of these interfaces have their dedicated tools for development and
testing.
The management of all the different tools can be quite tricky and directly
impacts the cost of maintenance. \\
These projects can have multiples teams that work on, such as business,
dev or testing.
Some teams usually don't have technical skills, like business or even the
testing team.
The communication between them is often hard and sometimes there is almost
no communication. \\
The methodology used, either Agile or Waterfall, does not matter.
Usually, an Agile project is more likely to follow good practices in
Quality and Testing but this is not necessarily the case.
The proposed solution is not Agile-oriented and will work on Waterfall
projects. \\
Finally, as Quality has a cost, low-quality projects usually don't have an
infinite budget.
Therefore the proposed solution only use community or free licences tools
and try to reuse at most as possible elements, tools, data etc.

\section{Justification}\label{sec:justification}
I have decided
