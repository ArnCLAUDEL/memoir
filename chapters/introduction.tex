This introduction will detail the problematic by defining what is
\textit{Quality} in a software project and what low-quality projects are
exposed to.
We will then define the relevant project contexts for this problematic.
Finally, we will justify the choice of this problematic by describing the
current context in my company and my personal experiences in the area of
Quality and Testing.

\section{Detailed Problematic}\label{sec:detailed-problematic}

%http://softwaretestingfundamentals.com/software-quality/
There are a lot of possible definitions for Quality in software projects, for
this problematic, we've chosen the one from the \textit{International Software
Testing Qualifications Board} :
\begin{quote}
The degree to which a component, system or process meets specified
requirements and/or user/customer needs and expectations.
\end{quote}
This definition is very abstract so we're going to detail what exactly is the
difference between a requirement, a need and an expectation. \\
%https://www.iiba.org/standards-and-resources/glossary/
%http://mohamedelgendy.com/blog/business-needs-vs-requirements.html
%https://westwindconsulting.com/Requirements%20vs%20expectations.html
According to the \textit{International Institute of Business Analysis}, a
requirement is a description of the need of a stakeholder that must be met in
order to achieve the business requirements.
They may serve as a bridge between business requirements and the various
categories of solution requirements.
A need is a problem of strategic importance to be addressed and could be
described as a high-level representation of the requirement. \\
These two terms refers to what the stakeholders express to the integration
team and it is (supposed to be) well defined.
However, an expectation is usually more unclear and harder to state because
it tends to be more subjective and is often implicit.
For example, a stakeholder might expect its application to run smoothly, no
matter how many users it has, or that sensitive data is well secured and
compliant with the current legislation.
He won't explicitly require that its application is reliable, fast, secure
and easy to adapt, because he could think that it's just the standard way to go.
\\
Quality details may have different shape and be at different level of the
project such as people interaction, process, tools and even production code. \\
\newline
Projects that don't pay enough attention to the details, will see their
technical debt growing bigger and bigger.
The consequence of high technical debt is generally a dramatic
cost increasing of every move in the project. \\
Let's show a quick example, we'll take a project that is meeting all the
requirements of the stakeholder and show some details (that don't prevent
the project features from working) that can slow down the project lifecycle
and occurs extra cost. \\
\newline
We have 1--2 business persons and 2--3 developers, the communication between
the two teams is weak and the needs/requirements are specified around
coffees, mails and sometimes on slides but nothing formal.
The developers interpret the feature and quickly implement it with an extensive
use of copy paste.
It's integrated and manually tested, also very quickly, and then the cycle
is repeated until all the features are implemented. \\
The business is then able to test the whole system and a lot of round trips
are made to correct the bugs and misunderstanding of the specs.
This part takes a lot longer than the development part, because developers
now need to dig again in the code and understand what they've written in
order to fix the bug.
They obviously forgot what they did in the first place, how they tested it
and they end up re creating the whole feature again.
In addition, the fix is likely to silently break other features, that were
previously tested and validated.
We can choose to run all the tests again for each round trip, but at a very
expensive cost or we can choose to accept the risk of introducing a bug into
production.
For now, we'll suppose that there's no new bug.
After this testing phase, all requirements are validated, the application is
delivered and all these steps will be repeated again for the next incoming
version. \\
\newline
At this point, we have a working software, but how about the quality ? \\
If we refer to our definition, a high quality software should meet the needs,
requirements and expectations of the stakeholder.
Needs and requirements are fulfilled, but what about expectations ? \\
Why is there so many bug or divergence with the specification ? \\
Why does it take so long to modify a small behaviour present in almost every
features ? \\
Why shall we run already passed tests again when another feature is modified
? \\
Why is it so hard to fix a bug whereas the feature was written very quickly ? \\
Where are the proof of each tests ? \\
Which test have been run ?
And when ?
On which version ? \\
\newline
All these kind of expectations, that one could expect for any project, are
not mandatory to make a working software.
Therefore, they can be classified as quality details and omitted on certain
projects, because they don't directly add business value to the software.
The problem is that every ignored details will slowly and constantly make
things harder and more costly overall. \\
\newline
This memoir will present the state of the art in term of tool
and process that help improving the quality of a project.
We'll highlight the residual problems and propose a solution to address them.

\section{Relevant Context}\label{sec:relevant-context}
This section describes which contexts are relevant to this problematic
and where the proposed solution is suitable. \\
This problematic is common on non-critical IT projects using
programming languages such as Java, Javascript, Python etc.
By non-critical, we reference any project where there is no human life,
financial market, aerospace system and any other "critical domain" at
the core of the application. \\
There are often a Web UI and an exposed API for the end users.
Each of these interfaces have their dedicated tools for development and
testing.
The management of all the different tools can be quite tricky and
directly impacts the cost of maintenance. \\
These projects can have multiples teams that work on, such as business,
dev or testing.
Some teams usually don't have technical skills, like business or even the
testing team.
The communication between them is often hard and sometimes there is
almost no communication. \\
The methodology used, either Agile or Waterfall, does not matter.
Usually, an Agile project is more likely to follow good practices in
Quality and Testing but this is not necessarily the case.
The proposed solution is not Agile-oriented and will work on Waterfall
projects. \\
Finally, as Quality has a cost, low-quality projects usually have smaller
budget.
Therefore the proposed solution only use community or free licences tools
and try to reuse at most as possible the same elements, tools, data etc.

\section{Justification}\label{sec:justification}
This section explains why I have chosen this problematic by describing
my current project, important experiences and my personal interests. \\
I've been working at Sopra Steria for one year, on the project
Chorus Pro, for the Agence pour l'Informatique Financi\`{e}re de l'Etat
(AIFE).
I was in internship in 2018 and I started my apprenticeship in the exact
same team.
It's a big project that handles digital invoices sent by companies to the
State.
It's mostly written in Java, with a Web Portal, REST / Soap API, EDI
exchanges and Spring Batch.
There are a lot of modules, components, environments, teams and of course
a lot of legacy code. \\
Until very recently, there was almost no unit / integration tests, it was
barely manual testing in dev environments and manual functional testing
in staging environment.
The consequences of the lack of testing were obvious and I realized why
testing is so important, the time it can save or the bugs it can prevent.
I saw some people reluctant to add new tests, to automate them or to
improve code quality and it appears that it really is a vicious circle.
\\
\newline
Beside my project where I understood why Quality and Testing are
important, I also attended to, in my opinion the best class of my
training, the Agile class by M. Zam.
This class was a very broad overview of methods, techniques, tools and a
lot of buzzwords that help to create better software.
I would rename this class as Software Craftsmanship, because it is a
bunch of good practices and real life complex methods that are usually
used only by Software Crafters. \\
I was very sceptical by what he showed us, because it seemed very complex
and not natural at all.
Things such as TDD, Pair Programming, DDD were completely different of
what I was used to.
I couldn't understand why people were talking so much about that, how can
people could even work with these methods. \\
So I tried to see what others do, and the best way I found was Meetup.
Meetup are small events, usually with several talks but sometimes you
simply code with other people and you make some tricky exercises or
discover new technologies.
I realized that these unusual methods were actually very common and that
it only depends on you to work with the people you want to. \\
\newline
This problematic was perfectly suited for me. \\
First of all, I'm very interesting in these Quality and Testing concerns,
so I can learn new things, methods, tools about it and see how common
problems are tackled. \\
Second, I was a bit tired of working on the same thing in my project and
I heard that they wanted to work on test automation.
So this problematic could help me to switch to a fresh mission that keeps
me in the mood. \\
Finally, I think that it will be very helpful for me later, in my future
job.
I can learn a lot of new methods and technologies that are used in the
real world and be better prepared to work with.
