This introduction will detail the problematic and show why there are
still projects with a low level of quality with almost no test.
We will also define the relevant contexts where the proposed solution
of this problematic is suitable.
Finally, I will justify the choice of this problematic by describing the
current context in my company and my personal experiences in the area of
Quality and Testing.
\section{Detailed Problematic}\label{sec:detailed-problematic}
There is a wide variety of IT projects, with different programming
languages, methodologies and business domains.
Practices change frequently over time, new technologies arise,
some disappear after several years, some are broadly used and are even older
than the programmers that use them. \\
Quality and Testing concerns are not new.
Prior to 2000, there was already projects with a very high level of quality
and strong automated tests, but it was very uncommon at this time.
In the early 2000's, a lot of new methodologies and tools emerged in order
to solve problems that programmers used to encounter. \\
In the 2010's, these methods and tools became mature and were in a
lot of projects.
More and more blogs, conferences or training were created to present them,
to show how to use them and introduce them in a legacy project.
The community grew and now a day they are almost a standard in new mainstream
projects. \\
\newline
However, even with the great enthusiasm of the community, there are still
projects where programmers, managers and clients are not convinced by
these methodologies and technologies and keep doing it the old way. \\
Poor design, no refactoring nor code reviews, no unit testing, minimum
manual functional tests, heterogeneous environments etc.
The list can be very long, so do for the consequences.
In the worst case, people give up and accept these bad practices so they
integrate them as part of the normal process. \\
Sometimes, it's hard to see the value of a tool that, at first glance, only
cost a lot of
efforts, time and therefore money, for no extra business value.
The benefits in a long run are generally ignored or under estimated.
The change of the team's habits does have a cost, requires new trainings,
new processes, some extra management for the new methodologies and
extra maintenance for the new tools.
But this is a necessary mean, an investment for the long run in order to
increase the overall quality and prevent the project from disaster. \\
\newline
The solution of this problematic starts therefore by correctly understanding
the features of existing methods and tools and then show how to easily
integrate them into a project.

\section{Relevant Context}\label{sec:relevant-context}
This section describes which contexts are relevant to this problematic and
where the proposed solution is suitable. \\
This problematic is common on non-critical IT projects using
programming languages such as Java, Javascript, Python etc.
By non-critical, we reference any project where there is no human life,
financial market, aerospace system and any other "critical domain" at the
core of the application. \\
There are often a Web UI and an exposed API for the end users.
Each of these interfaces have their dedicated tools for development and
testing.
The management of all the different tools can be quite tricky and directly
impacts the cost of maintenance. \\
These projects can have multiples teams that work on, such as business,
dev or testing.
Some teams usually don't have technical skills, like business or even the
testing team.
The communication between them is often hard and sometimes there is almost
no communication. \\
The methodology used, either Agile or Waterfall, does not matter.
Usually, an Agile project is more likely to follow good practices in
Quality and Testing but this is not necessarily the case.
The proposed solution is not Agile-oriented and will work on Waterfall
projects. \\
Finally, as Quality has a cost, low-quality projects usually have smaller
budget.
Therefore the proposed solution only use community or free licences tools
and try to reuse at most as possible the same elements, tools, data etc.

\section{Justification}\label{sec:justification}
This section explains why I have chosen this problematic by describing my
current project, important experiences and my personal interests. \\
I've been working at Sopra Steria for one year, on the project Chorus Pro,
for the Agence pour l'Informatique Financi\`{e}re de l'Etat (AIFE).
I was in internship in 2018 and I started my apprenticeship in the exact
same team.
It's a big project that handles digital invoices sent by companies to the
State.
It's mostly written in Java, with a Web Portal, REST / Soap API, EDI
exchanges and Spring Batch.
There are a lot of modules, components, environments, teams and of course
a lot of legacy code. \\
Until very recently, there was almost no unit / integration tests, it was
barely manual testing in dev environments and manual functional testing
in staging environment.
The consequences of the lack of testing were obvious and I realized why
testing is so important, the time it can save or the bugs it can prevent.
I saw some people reluctant to add new tests, to automate them or to improve
code quality and it appears that it really is a vicious circle. \\
\newline
Beside my project were I understood why Quality and Testing are important,
I also attended to, in my opinion the best class of my training, the Agile
class by M. Zam.
This class was a very broad overview of methods, techniques, tools and a lot
of buzzwords that help to create better software.
I would rename this class as Software Craftsmanship, because it is a bunch
of good practices and real life complex methods that are usually used only
by Software Crafters. \\
I was very sceptical by what he showed us, because it seemed very complex
and not natural at all.
Things such as TDD, Pair Programming, DDD were completely different of what
I was used to.
I couldn't understand why people were talking so much about that, how can
people could even work with these methods. \\
So I tried to see what others do, and the best way I found was Meetup.
Meetup are small events, usually with several talks but sometimes you simply
code with other people and you make some tricky exercises or discover new
technologies.
I realized that these unusual methods were actually very common and that it
only depend on you to work the people you want to. \\
\newline
This problematic was perfectly suited for me. \\
First of all, I'm very interesting in these Quality and Testing concerns,
so I can learn new things, methods, tools about it and see how common
problems are tackled. \\
Second, I was a bit tired of working on the same thing in my project and
I heard that they wanted to work on test automation.
So this problematic could help me to switch to a fresh mission that keeps me
in the mood. \\
Finally, I think that it will be very helpful for me later, in my future
job.
I can learn a lot of new methods and technologies that are used in the real
world and be better prepared to work with.
